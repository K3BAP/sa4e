% !TeX spellcheck = de_DE
\documentclass[12pt]{article}
\usepackage[utf8]{inputenc}
\usepackage{geometry}
\usepackage{svg}
\usepackage{float}
\usepackage{caption}
\usepackage{amsmath,amsthm,amsfonts,amssymb,amscd}
\usepackage{fancyhdr}
\usepackage{titlesec}
\usepackage{hyperref}
\usepackage{listings}
\usepackage[skip=3pt]{parskip}
\usepackage[ngerman]{babel}
\pagestyle{empty}
\titleformat*{\section}{\large\bfseries}
\titleformat*{\subsection}{\bfseries}

%
\geometry{
	a4paper,
	total={170mm,240mm},
	left=20mm,
	top=30mm,
}

\date{}
%Bitte ausfüllen
\newcommand\course{Software Architectures for Enterprises}
\newcommand\hwnumber{\large Portfolio 2}
\newcommand\Name{Fabian Sponholz}
\newcommand\Neptun{1561546}

%Matheinheiten
\newcommand\m{\:\textrm{m}}
\newcommand\M{\:\Big[\textrm{m}\Big]}
\newcommand\mm{\:\textrm{mm}}
\newcommand\MM{\:\Big[\textrm{mm}\Big]}
\newcommand\un{\underline}
\newcommand\s{\:\textrm{s}}
\newcommand\bS{\:\Big[\textrm{S}\Big]}
\newcommand\ms{\:\frac{\textrm{m}}{\textrm{s}}}
\newcommand\MS{\:\Big[\frac{\textrm{m}}{\textrm{s}}\Big]}
\newcommand\mss{\:\frac{\textrm{m}}{\textrm{s}^2}}
\newcommand\MSS{\:\Big[\frac{\textrm{m}}{\textrm{s}^2}\Big]}

%Trennlinie
\newcommand\separator{\rule{\linewidth}{0.5pt}}

%Bitte nicht einstellen
\renewcommand{\figurename}{Abbildung}
\renewcommand{\tablename}{Tabelle}
\pagestyle{fancyplain}
\headheight 35pt
\lhead{\Name\\\Neptun}
\chead{\textbf{ \hwnumber}}
\rhead{\course \\ \today}
\lfoot{}
\cfoot{}
\rfoot{\small\thepage}
\headsep 1.5em

\begin{document}
	
\section*{Aufgabe 1 - Modellierung}
\subsection*{Zu erwartendes Lastaufkommen}
Bei der Berechnung des erwarteten Lastaufkommens in der Vorweihnachtszeit gehen wir vereinfacht davon aus, dass es auf der Welt etwa 5 Milliarden Kinder und kindgebliebene Erwachsene gibt, die ihre Wunschliste an Santa Claus übermitteln möchten.
Außerdem gehen wir davon aus, dass all diese Einsendungen innerhalb von 90 Tagen (im letzten Quartal) eingehen und jede Person nur \emph{einen} Wunschzettel, wenn auch mit mehreren Wünschen, einsendet.

Dabei wurde vereinfachend die Annahme gemacht, dass die gesamte Menschheit den Weihnachtsmann als religions- und kulturunabhängige Instanz respektiert und eine fristgerechte Lieferung der Geschenke an dem sich historisch in einer seit Jahrtausenden fortlaufenden Studie als am günstigsten herausgestellten Liefertermin (dem 24.12.) gewohnt ist.
Maßgeblich bei der Berechnung des Liefertermins ist das zu diesem Zeitpunkt vorliegende Gleichgewicht zwischen Gemütlichkeit des Abends (nimmt mit Fortschreiten des Winters zu) und Wahrscheinlichkeit einer schneesturmfreien Nacht (nimmt mit Fortschreiten des Winters ab).
Dabei wird entgegen der Realität davon ausgegangen, dass der Jahreszeitenzyklus überall auf der Erde synchron verläuft.

Wie dem auch sei, das System durchläuft jedes Jahr drei Phasen, die je ein wachsendes Anfrageaufkommen im \emph{XMasWishes}-System erwarten lassen:

\begin{itemize}
	\item \textbf{Phase 1: Wunschlisten-Einsammel-Phase} - 90 Tage vor dem Stichtag wird das System auf öffentlich geschaltet und ist bereit für eingehende Wunschlisten. Dabei wird innerhalb dieser Zeit etwa 5 Milliarden mal schreibend auf das System zugegriffen. Die Anfragefrequenz in Hertz berechnet sich wie folgt: 
	$$freq_{collection} = 5000000000 \div (90 \cdot 24 \cdot 60 \cdot 60) hz \approx 650 hz $$
	
	\item \textbf{Phase 2: Produktionsphase} - In den letzten 30 Tagen vor dem Stichtag produzieren die Elfen alle Geschenke. Hierbei fragt jeder Elf aus dem System einen Wunsch ab, der noch nicht bearbeitet wurde, um diesen dann umzusetzen. Der Wunsch wird dann auf \emph{In Bearbeitung} gesetzt. Hier wird jeder Wunsch genau ein mal abgefragt. Bei durchschnittlich 3 Wünschen pro Wunschzettel ergibt sich das Anfrageaufkommen wie folgt:
	$$freq_{production} = 3 \cdot 5000000000 \div (30 \cdot 24 \cdot 60 \cdot 60) hz \approx 5800 hz$$
	
	\item \textbf{Phase 3:Auslieferungsphase} - Am Stichtag lädt der Weihnachtsmann alle Geschenke auf seinen Schlitten und setzt den Zustand aller Geschenke von \emph{In Bearbeitung} auf \emph{In Zustellung}. Dies ist mit einem einzelnen Aufruf möglich.
	Danach ruft er während der Auslieferung jede Wunschliste noch einmal ab, um in jedem Haus die passenden Geschenke abzuliefern.
	Wem unrealistisch erscheint, dass der Weihnachtsmann alle Geschenke selbst ausliefert, dem sei gesagt, dass der Weihnachtsmann sich mithilfe modernster Quantentechnologie stets in einer Superposition befindet und somit an mehreren Orten gleichzeitig sein kann.
	Es wird also innerhalb eines Tages jede Wunschliste einmal in einem Aufruf abgerufen und die entsprechenden Geschenke danach als ausgeliefert markiert.
	Das Anfrageaufkommen ergibt sich dann wie folgt:
	$$freq_{delivery} = 2 \cdot 5000000000 \div (1 \cdot 24 \cdot 60 \cdot 60) hz \approx120000 hz$$
	
\end{itemize}

\subsection*{Modellierung einer skalierbaren Architektur für XMasWishes}
Folgender Ansatz wird in diesem Projekt umgesetzt:

\begin{itemize}
	\item \textbf{Backend:} Horizontal skalierbare, verteilte NoSQL-Datenbank. Simuliert eine Relationale Datenbank und kann in einer SQL-Ähnlichen Syntax angesprochen werden.
	\item \textbf{Business-Logik:} Stateless Service, der die Schnittstelle zwischen Nutzer und Datenbank herstellt. Kann beliebig oft repliziert werden.
	\item \textbf{Load-Balancer} Wird zwischen Frontend und Business-Logik geschaltet, um die Last gleichmäßig auf die Services zu verteilen.
	\item \textbf{Frontend} Web-Applikation im Browser des Nutzers, ggf. auf dem gleichen Server wie die Business-Logik gehostet.
\end{itemize}

Ein Authentifizierungsservice wie OAuth wird nicht implementiert, da der Nordpol bereits über eine quantenbasierte Firewall verfügt, die sicherstellt, dass jeder nur auf das zugreifen kann, was er benötigt.

\section*{Aufgabe 2 - Konkretisierung}

\begin{figure}[H]
	\centering
	\includegraphics[width=0.8\textwidth]{./img/architecture}
	\caption{Architektur des Systems}
\end{figure}



	
\end{document}