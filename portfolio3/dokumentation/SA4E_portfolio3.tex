% !TeX spellcheck = de_DE
\documentclass[12pt]{article}
\usepackage[utf8]{inputenc}
\usepackage{geometry}
\usepackage{svg}
\usepackage{float}
\usepackage{caption}
\usepackage{amsmath,amsthm,amsfonts,amssymb,amscd}
\usepackage{fancyhdr}
\usepackage{titlesec}
\usepackage{hyperref}
\usepackage{listings}
\usepackage[skip=3pt]{parskip}
\usepackage[ngerman]{babel}
\pagestyle{empty}
\titleformat*{\section}{\large\bfseries}
\titleformat*{\subsection}{\bfseries}

%
\geometry{
	a4paper,
	total={170mm,240mm},
	left=20mm,
	top=30mm,
}

\date{}
%Bitte ausfüllen
\newcommand\course{Software Architectures for Enterprises}
\newcommand\hwnumber{\large Portfolio 3}
\newcommand\Name{Fabian Sponholz}
\newcommand\Neptun{1561546}

%Matheinheiten
\newcommand\m{\:\textrm{m}}
\newcommand\M{\:\Big[\textrm{m}\Big]}
\newcommand\mm{\:\textrm{mm}}
\newcommand\MM{\:\Big[\textrm{mm}\Big]}
\newcommand\un{\underline}
\newcommand\s{\:\textrm{s}}
\newcommand\bS{\:\Big[\textrm{S}\Big]}
\newcommand\ms{\:\frac{\textrm{m}}{\textrm{s}}}
\newcommand\MS{\:\Big[\frac{\textrm{m}}{\textrm{s}}\Big]}
\newcommand\mss{\:\frac{\textrm{m}}{\textrm{s}^2}}
\newcommand\MSS{\:\Big[\frac{\textrm{m}}{\textrm{s}^2}\Big]}

%Trennlinie
\newcommand\separator{\rule{\linewidth}{0.5pt}}

%Bitte nicht einstellen
\renewcommand{\figurename}{Abbildung}
\renewcommand{\tablename}{Tabelle}
\pagestyle{fancyplain}
\headheight 35pt
\lhead{\Name\\\Neptun}
\chead{\textbf{ \hwnumber}}
\rhead{\course \\ \today}
\lfoot{}
\cfoot{}
\rfoot{\small\thepage}
\headsep 1.5em

\begin{document}
	
\section*{Aufgabe 1 - Basis-Architektur}
\subsection*{Erste Überlegungen zur Architektur}
Zuerst habe ich mir überlegt, wie ich das Spiel \emph{Ave Cäsar} auf eine Architektur aus Microservices und einer skalierbaren Datenverwaltung mit Apache Kafka abbilden kann.
Dabei liegt es nahe, in Kafka zunächst für jedes Segment ein eigenes Topic anzulegen, für das dann im Folgenden verschiedene Producer und Consumer erstellt werden können.

Die Streitwagen werden dann jeweils durch eine Message repräsentiert, optimalerweise in Form eines JSON-Strings um Statusinformationen zu transportieren.
 

Jedes Segment wird dann von einem eigenen Microservice repräsentiert, der jeweils als Consumer auf dem ihm zugeordneten Kafka Topic agiert.
So werden ankommende Streitwagen vom Service gefunden und können an das jeweils nächste Segment (bzw. eines der möglichen nächsten Segmente) durch einen Producer weitergeleitet werden, wo bereits der nächste Segment-Service bereitsteht.

Für das Management des Systems muss außerdem ein zentrales Verwaltungsprogramm geschrieben werden, das den Streckenverlauf aus dem zuvor generierten JSON einliest und dann folgende Schritte unternimmt:

\begin{enumerate}
	\item \emph{Setup des Kafka-Servers} - Für jedes Streckensegment und für das Time-Tracking muss ein Kafka-Topic erstellt werden.
	\item \emph{Instantiierung der Microservices} - Für jedes Segment muss eine Instanz des Segment-Service erstellt werden, wobei die nötigen Informationen an das Programm weitergegeben werden müssen.
	\item \emph{Start-and-Goal Segmente} - Mit den Start-And-Goal-Segmenten muss in besonderer Weise kommuniziert werden, damit eine Ausgabe der Gesamtlaufzeit am Ende möglich ist.
	Es ist auch denkbar, die entsprechenden Programme erst bei Start des Rennens zu instantiieren.
	Ein zusätzliches Kafka-Topic sollte verwendet werden, um die Umlaufzeiten zu messen (es ist ja möglich, dass ein Wagen auf einem anderen Segment ankommt als jenes, von dem er gestartet ist).
	\item \emph{Start des Rennens} - Die Start-and-Goal Segmente müssen auf Kommando informiert/instantiiert werden, um die Streitwagen entsprechend ins Rennen zu schicken.
	\item \emph{Ausgabe der Ergebnisse} - Am Ende müssen die Umlaufzeiten der einzelnen Wagen ausgegeben werden.
	\item \emph{Aufräumen} - Die erstellten Prozesse müssen terminiert und die Kafka Topics gelöscht werden.
\end{enumerate}


\end{document}